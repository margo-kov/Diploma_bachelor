\chapter{Conclusion}

The work aimed to reduce the time required for molecular docking by excluding from the time-consuming molecular docking
the molecules which are less prone to binding to a target.
The work has been successful: the designed algorithm finds 60\% of top-1\% molecules after screening only 10\% of the library, and more than 80s\% -- after screening 25\%.
More detailed results of this work are:

\begin{itemize}
    \item Performance of regressors and classifiers trained on the Morgan fingerprints with default parameters have been analysed.
    Results indicate that higher recall was achieved when working with regressors.
    \item Different parameters (radius, size) for the Morgan fingerprint have been utilized in predictions.
    Best recall have showed regressors trained on Morgan fingerprints with default parameters (radius=2 and size=2048).
    \item Morgan fingerprints with default parameters have been compared with atom pairs fingerprints.
    Models trained on these fingerprints have had comparable results, so further work have been done using Morgan fingerprints with radius=2 and size=2048.
    \item Iterative algorithms with various approaches to train set augmentation ("add" or "noadd") and complex model creation ("LastModel", "MeanRank", "TopFromEveyModel") have been accessed.
    The algorithm with "MeanRank" complex model and "noadd" train set augmentation approach has appeared to obtain the highest fraction of discovered hits.
    \item The percentage of hits discovered by the iterative algorithm with linear regression as a single model has been evaluated for all variations of the algorithm depending on the docking batch size.
	The batch size 8,000 was found to have the best performance.
    \item Iterative algorithm has been compared to an independent docking run.
    The number of hits discovered with 4-fold reduction in time compared to exhaustive docking has been estimated as 80\%, with 10-fold reduction - 60\%.
\end{itemize}
